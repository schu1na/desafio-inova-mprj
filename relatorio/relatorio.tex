\documentclass[a4paper,12pt]{article}
\usepackage[utf8]{inputenc}
\usepackage[brazil]{babel}
\usepackage{graphicx} % Para imagens
\usepackage{geometry}
\geometry{a4paper, left=3cm, top=3cm, right=2cm, bottom=2cm}
\usepackage{hyperref}
\usepackage{float}
\hypersetup{
    colorlinks=true,       % True: links coloridos; False: caixas coloridas
    linkcolor=black,        % Cor dos links internos (como no Sumário)
    filecolor=magenta,      
    urlcolor=blue,          % Cor de links externos (URLs)
    citecolor=black,
    pdftitle={Relatório de Análise de Despesas Públicas},
}
\usepackage{helvet}
\renewcommand{\familydefault}{\sfdefault}



\begin{document}

% Capa do relatório
\begin{titlepage}
    \begin{center}
        \vspace*{1cm}
        
        \Huge
        \textbf{Análise de Despesas Públicas}
        
        \vspace{0.5cm}
        \LARGE
        Diagnóstico de Inconsistências e Fluxos Orçamentários
        
        \vspace{1.5cm}
        
        \textbf{Relatório Técnico}
        
        \vfill
        
        \line(1,0){400} \\
        \large
        \textbf{Candidato:} Gabriel Santos Schuina \\
        \small \href{https://github.com/schu1na/desafio-inova-mprj}{\texttt{github.com/schu1na/desafio-inova-mprj}} \\
        \textit{Processo Seletivo - Inova MPRJ} \\
        \today
        
    \end{center}
\end{titlepage}

% Sumário
\tableofcontents
\newpage

\begin{abstract}
\noindent
Este relatório técnico apresenta os resultados da auditoria de dados orçamentários, 
fundamentada no fluxo legal da despesa pública. A partir da análise de 500 empenhos, 
buscou-se identificar inconformidades lógicas e indícios de irregularidades na execução 
financeira. Foram detectadas três inconsistências críticas: pagamentos superiores ao valor 
empenhado, inversão cronológica de etapas (pagamento anterior ao empenho) e pagamentos 
realizados sem a devida liquidação. Os achados apontam para falhas nos controles internos 
da gestão orçamentária.
\end{abstract}


\section{Introdução}
A execução da despesa pública no Brasil obedece a um procedimento rígido estipulado pela 
Lei 4.320/64, estruturado em três estágios fundamentais e sequenciais: o Empenho (reserva 
de dotação), a Liquidação (reconhecimento do direito do credor) e o Pagamento (desembolso 
financeiro). A estrita observância dessa ordem cronológica e lógica é vital para a 
transparência e a conformidade fiscal.

O presente relatório tem como objetivo auditar a integridade de uma base de dados contendo 
500 registros de despesas, verificando a consistência entre essas etapas. Através de técnicas 
de análise de dados, buscou-se identificar violações das regras de 
negócio que possam comprometer a confiabilidade da execução orçamentária.

A seguir, detalha-se a metodologia utilizada e as inconsistências diagnosticadas durante o 
processo de validação.

\section{Diagnóstico de Inconsistências}

\subsection{Violação de Teto: Pagamentos Superiores ao Empenho}

O empenho de despesa, conforme estipulado pela legislação financeira, atua como um teto limitador 
para a execução orçamentária, criando a obrigação de pagamento apenas até o montante reservado. 

No entanto, a análise dos dados revelou a existência de \textbf{225} registros onde essa lógica foi violada, 
com valores pagos excedendo o saldo empenhado. O valor total pago a mais chega a \textbf{R\$3.500.000,00}. 
Essa inconsistência representa um risco crítico de controle, pois indica saídas de caixa (pagamentos) 
sem o devido planejamento orçamentário. A ausência de uma trava sistêmica para impedir tal operação sugere 
vulnerabilidade no sistema de gestão financeira analisado.


\begin{figure}[H]
    \centering
    \includegraphics[width=0.75\textwidth]{img/pagamentos_maior_empenho.jpg}
    \caption{Distribuição dos fornecedores e dos objetos de serviço com pagamentos maiores que o valor do empenho.}
\end{figure}


\subsection{Quebra de Fluxo: Pagamentos sem Liquidação Prévia}
A liquidação é o estágio da despesa onde se verifica o direito adquirido do credor, tendo por base títulos e 
documentos comprobatórios do respectivo crédito (Art. 63 da Lei 4.320/64). Em termos práticos, é a etapa de 
conferência da entrega do bem ou serviço.

A análise dos dados apontou a ocorrência de \textbf{40} pagamentos realizados sem o registro de liquidação 
correspondente, os quais somam \textbf{R\$985.500,00}. Esta quebra de fluxo é crítica, pois sugere que houve desembolso 
financeiro sem a validação formal de que a contrapartida foi entregue à administração pública. A inexistência 
desta etapa compromete a rastreabilidade da despesa e expõe o erário ao risco de pagamentos indevidos por serviços 
não prestados.

\begin{figure}[H]
    \centering
    \includegraphics[width=0.75\textwidth]{img/quebra_fluxo.jpg}
    \caption{Distribuição dos objetos de serviço com pagamentos sem liquidação.}
\end{figure}


\subsection{Inversão Cronológica: Pagamentos Anteriores ao Empenho}

O artigo 60 da Lei 4.320/64 é taxativo ao determinar que "é vedada a realização de despesa sem prévio empenho". 
O empenho é o ato que cria a obrigação para o Estado; portanto, cronologicamente, ele deve preceder qualquer 
desembolso.

A análise detectou um agrupamento anômalo de \textbf{41 registros} onde a data de pagamento é anterior à data de 
emissão do empenho. Ao aprofundar a investigação sobre esses casos, identificou-se um padrão temporal exato: em 
todas as 41 ocorrências, o pagamento foi registrado precisamente \textbf{5 dias antes} do respectivo empenho.

A regularidade desse intervalo (delta $t = -5$ dias) sugere fortemente a existência de um erro sistêmico ou de um 
procedimento automatizado de lançamento de dados que está operando com datas defasadas, e não meros erros manuais 
aleatórios de digitação.

\begin{figure}[H]
    \centering
    \includegraphics[width=0.75\textwidth]{img/pagamentos_antes_empenho.jpg}
    \caption{Distribuição dos objetos de serviço com pagamentos anteriores ao empenho.}
\end{figure}

\section{Conclusão e Recomendações}

A auditoria realizada na base de dados de execução orçamentária revelou inconsistências graves que comprometem a 
conformidade fiscal e a transparência dos gastos públicos. O somatório das irregularidades detectadas — incluindo 
pagamentos excedentes (R\$ 3,5 mi) e desembolsos sem liquidação (R\$ 985 mil) — aponta para uma fragilidade nos 
controles preventivos do sistema atual.

Mais do que erros operacionais isolados, a padronização das falhas (como o intervalo fixo de 5 dias na inversão 
cronológica) sugere a existência de *bugs* em rotinas automatizadas ou falhas na parametrização do software de 
gestão financeira (Siafe/ERP).

Diante deste cenário, recomendam-se as seguintes ações corretivas imediatas:

\begin{itemize}
    \item \textbf{Implementação de Travas Lógicas (Hard Blocks):} Configurar o banco de dados para rejeitar 
    automaticamente qualquer transação de pagamento cujo valor supere o saldo de empenho disponível 
    ($V_{pgto} > V_{empenho}$) ou cuja data seja anterior à do empenho.
    \item \textbf{Revisão de Scripts de Automação:} Auditar os *jobs* de processamento de dados para corrigir 
    o *delay* negativo de 5 dias identificado nos 41 registros anômalos.
    \item \textbf{Saneamento da Base:} Realizar um pente-fino nos 40 casos de pagamento sem liquidação para 
    verificar se houve a efetiva prestação do serviço, mitigando risco de dano ao erário.
\end{itemize}

Esta análise demonstra que, embora os dados estejam disponíveis, a integridade da informação requer monitoramento contínuo e regras de validação mais estritas na entrada dos dados.
\end{document}